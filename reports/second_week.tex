\documentclass{article}
\usepackage{graphicx}
\usepackage{epstopdf}
\usepackage{amsmath}
\usepackage{listings}


\begin{document}
\title{LAB ROTATION 02 \\ 2nd Week Assignment }
\date{18.09.2013}
\author{\c{S}eyma Bayrak, Advisor: Philipp H\"{o}vel}
\maketitle

\section{Nonlinear Dynamics of Neural Networks}
\subsection{Simple FitzHugh-Nagumo Model}
This section is a continuing part of \textit{1st Week Assignment}, it aims to analyze the effect of parameters $\varepsilon$ and $\gamma$ in the equations (1) and (2) given below and to plot trajectories on nullcline graphs. 
\begin{equation}
 \varepsilon \dot{x} = x- \frac{x^3}{3}-y 
\end{equation}
\begin{equation}
 \dot{y}=x+a 
\end{equation} 

\begin{center}
\includegraphics[width=\textwidth]{xt_yt_a_130_eps_0005.eps}
% tau4_M1.eps: 0x0 pixel, 300dpi, 0.00x0.00 cm, bb= -304   -42   918   834
\begin{footnotesize}
 Figure 1, $a=1.30$, $\varepsilon=0.005$, $(x_0,y_0)=(-0.05,-0.75)$
\end{footnotesize}
\end{center}

\begin{center}
\includegraphics[width=\textwidth, height=50mm]{trajectory_a_130_eps_0005.eps}
% tau4_M1.eps: 0x0 pixel, 300dpi, 0.00x0.00 cm, bb= -304   -42   918   834
\begin{footnotesize}
 Figure 2, $a=1.30$, $\varepsilon=0.005$, $(x_0,y_0)=(-0.05,-0.75)$
\end{footnotesize}
\end{center}

\begin{center}
\includegraphics[width=\textwidth]{xt_yt_a_130_eps_1.eps}
% tau4_M1.eps: 0x0 pixel, 300dpi, 0.00x0.00 cm, bb= -304   -42   918   834
\begin{footnotesize}
 Figure 3, $a=1.30$, $\varepsilon=1$, $(x_0,y_0)=(-0.05,-0.75)$
\end{footnotesize}
\end{center}

\begin{center}
\includegraphics[width=\textwidth, height=55mm]{trajectory_a_130_eps_1.eps}
% tau4_M1.eps: 0x0 pixel, 300dpi, 0.00x0.00 cm, bb= -304   -42   918   834
\begin{footnotesize}
 Figure 4, $a=1.30$, $\varepsilon=1$, $(x_0,y_0)=(-0.05,-0.75)$
\end{footnotesize}
\end{center}

\begin{itemize}

 \item $a$ effect: Bifurcation anaylsis done in 1st Week Assignment showed that, as long as $|a|>1$, the system is expected to be stable. Figures 1-4 have $a=1.30$ and they are all stable, or in other words "excitable".

 \item $\varepsilon$ effect: it does not affect stability but plays a role in the time evolution and pathway of initial points $x_0$ and $y_0$ - how long it takes initial point to reach to the stable point. When $\varepsilon$ is small, $x$ and $y$ reaches stability faster. (?)
\end{itemize}

\begin{center}
\includegraphics[width=\textwidth]{xt_yt_a_090_eps_0005.eps}
% tau4_M1.eps: 0x0 pixel, 300dpi, 0.00x0.00 cm, bb= -304   -42   918   834
\begin{footnotesize}
 Figure 5, $a=0.90$, $\varepsilon=0.005$, $(x_0,y_0)=(-0.05,-0.75)$
\end{footnotesize}
\end{center}

\begin{center}
\includegraphics[width=\textwidth, height=55mm]{trajectory_a_090_eps_0005.eps}
% tau4_M1.eps: 0x0 pixel, 300dpi, 0.00x0.00 cm, bb= -304   -42   918   834
\begin{footnotesize}
 Figure 6, $a=0.90$, $\varepsilon=0.005$, $(x_0,y_0)=(-0.05,-0.75)$
\end{footnotesize}
\end{center}

\begin{center}
\includegraphics[width=\textwidth]{xt_yt_a_090_eps_1.eps}
% tau4_M1.eps: 0x0 pixel, 300dpi, 0.00x0.00 cm, bb= -304   -42   918   834
\begin{footnotesize}
 Figure 7, $a=0.90$, $\varepsilon=1$, $(x_0,y_0)=(-0.05,-0.75)$
\end{footnotesize}
\end{center}

\begin{center}
\includegraphics[width=\textwidth, height=55mm]{trajectory_a_090_eps_1.eps}
% tau4_M1.eps: 0x0 pixel, 300dpi, 0.00x0.00 cm, bb= -304   -42   918   834
\begin{footnotesize}
 Figure 8, $a=0.90$, $\varepsilon=1$, $(x_0,y_0)=(-0.05,-0.75)$
\end{footnotesize}
\end{center}
\begin{itemize}

 \item $a$ effect: When $|a|<1$, the system is unstable. Figures 5 and 7 have oscillations with $a=0.90$. 

 \item $\varepsilon$ effect: It affects the period of oscillations, when $\varepsilon$ is small, oscillatory behavior of $x$ and $y$ happens more frequently. (What about Figure 6 and 8?)
\end{itemize}

\subsection{Extended FitzHugh-Nagumo Model}

\begin{equation}
 \varepsilon \dot{x} = x- \frac{x^3}{3}-y 
\end{equation}
\begin{equation}
 \dot{y}=x+a -\gamma y
\end{equation}

Fixed point can be now \textit{stable, unstable, saddle.} \newline
Saddle point means that the signs of real parts of the eigenvalues of Jacobian matrix are different. We would like to eliminate saddle points.(?)

\[
\textbf{J}=
\left( {\begin{array}{cc }
(1-x_f)^2  &  -1  \\
\varepsilon &  -\varepsilon\gamma  \\

\end{array} } \right)
\]

\begin{equation*}
 \lambda_{1,2}=\frac{trJ\pm \sqrt{tr^2J-4 det J}}{2}=
\frac{trJ \pm trJ(\sqrt{1-\frac{4detJ}{tr^2J}})}{2}
\end{equation*}

In order to keep the real part of eigenvalue to be dominated by the first term of the equation above ($trJ$), the term in square root must be either positive and smaller than 1 or it must be negative, which contributes to $\lambda$ only with a complex part. This is done by assuming $det(J)>0$. Then the sign of the eigenvalues is controlled directly by $trJ$, it is either positive for $\lambda_1$ and $\lambda_2$ or negative.

\begin{equation*}
det(J)=\varepsilon(\gamma (x_f^2-1)+1)>0 \Longrightarrow 0<\gamma<1 
 \end{equation*}

\begin{center}
\includegraphics[width=\textwidth]{xt_yt_a_130_eps_0005_gam_05.eps}
% tau4_M1.eps: 0x0 pixel, 300dpi, 0.00x0.00 cm, bb= -304   -42   918   834
\begin{footnotesize}
 Figure 9, $a=1.30$, $\varepsilon=0.005$, $\gamma=0.5$,  $(x_0,y_0)=(-0.75,-1)$ 
\end{footnotesize}
\end{center}

\begin{center}
\includegraphics[width=\textwidth, height=55mm]{trajectory_a_130_eps_0005_gam_05.eps}
% tau4_M1.eps: 0x0 pixel, 300dpi, 0.00x0.00 cm, bb= -304   -42   918   834
\begin{footnotesize}
 Figure 10,  $a=1.30$, $\varepsilon=0.005$,                   $\gamma=0.5$, $(x_0,y_0)=(-0.75,-1)$  
\end{footnotesize}
\end{center}

\begin{center}
\includegraphics[width=\textwidth]{xt_yt_a_130_eps_1_gam_05.eps}
% tau4_M1.eps: 0x0 pixel, 300dpi, 0.00x0.00 cm, bb= -304   -42   918   834
\begin{footnotesize}
 Figure 11, $a=1.30$, $\varepsilon=1$, $\gamma=0.5$,  $(x_0,y_0)=(-0.75,-1)$ 
\end{footnotesize}
\end{center}

\begin{center}
\includegraphics[width=\textwidth, height=50mm]{trajectory_a_130_eps_1_gam_05.eps}
% tau4_M1.eps: 0x0 pixel, 300dpi, 0.00x0.00 cm, bb= -304   -42   918   834
\begin{footnotesize}
 Figure 12,  $a=1.30$, $\varepsilon=0.005$,                   $\gamma=0.5$, $(x_0,y_0)=(-0.75,-1)$  
\end{footnotesize}
\end{center}

\begin{center}
\includegraphics[width=\textwidth]{xt_yt_a_090_eps_0005_gam_05.eps}
% tau4_M1.eps: 0x0 pixel, 300dpi, 0.00x0.00 cm, bb= -304   -42   918   834
\begin{footnotesize}
 Figure 13, $a=0.90$, $\varepsilon=0.005$, $\gamma=0.5$,  $(x_0,y_0)=(-0.75,-1)$ 
\end{footnotesize}
\end{center}

\begin{center}
\includegraphics[width=\textwidth, height=50mm]{trajectory_a_090_eps_0005_gam_05.eps}
% tau4_M1.eps: 0x0 pixel, 300dpi, 0.00x0.00 cm, bb= -304   -42   918   834
\begin{footnotesize}
 Figure 14,  $a=0.90$, $\varepsilon=0.005$, $\gamma=0.5$,  $(x_0,y_0)=(-0.75,-1)$  
\end{footnotesize}
\end{center}

\begin{itemize}

 \item $a$ effect: When $|a|<1$, the system was unstable in simple FHN model, however the stability in extended FHN model is now controlled by not only $a$ but also $\gamma$.  Figures 13 and 14 shows stability with $|a|<1$ and $\gamma=0.5$ in opposite to the Figures 5-8. 
 
\item $\varepsilon$ effect: It affects again the time period of initial points to reach to the stability and pathway of trajectory. Smaller $\varepsilon$ brings stability faster in Figure 13 compared to Figure 11. 
\end{itemize}

\begin{center}
\includegraphics[width=\textwidth]{xt_yt_a_090_eps_04_gam_005.eps}
% tau4_M1.eps: 0x0 pixel, 300dpi, 0.00x0.00 cm, bb= -304   -42   918   834
\begin{footnotesize}
 Figure 15, $a=0.90$, $\varepsilon=0.4$, $\gamma=0.05$,  $(x_0,y_0)=(-0.75,-1)$ 
\end{footnotesize}
\end{center}

\begin{center}
\includegraphics[width=\textwidth, height=55mm]{trajectory_a_090_eps_04_gam_005.eps}
% tau4_M1.eps: 0x0 pixel, 300dpi, 0.00x0.00 cm, bb= -304   -42   918   834
\begin{footnotesize}
 Figure 16,  $a=0.90$, $\varepsilon=0.04$, $\gamma=0.05$,  $(x_0,y_0)=(-0.75,-1)$  
\end{footnotesize}
\end{center}

\begin{itemize}

 \item $\gamma$ effect: This parameter controls the stability or instability of the system. We already assumed $\gamma$ between 0 and 1 as it was discussed in bifurcation analysis of extenden FHN model. When $\gamma$ is close to 0, then the system acts unstable and $x$-$y$ seem to oscillate. (?)  Figure 13 ($\gamma=0.5$) turns into a series of oscillatory behavior for $x$ and $y$ in Figure 15 ($\gamma=0.05$).  Here I did not necessariyl keep $\varepsilon$ at the same values, but we are already sure that $\varepsilon$ has nothing to do with stability.
\end{itemize}
\newpage

\section{Correlation Matrix of Functional Connectivity}

The provided data file \textit{A.txt} is an example of fMRI signals reflecting functional connectivities of a mammalian brain at resting state. \textit{A.txt} is a simple NxN matrix, where $N=64$, meaning that the brain is assumed to have $N=64$ functionally separated nodes. The connectivity matrix can be applied to a threshold value and then converted into a new matrix, which has 1 for the elements of matrix greater than the threshold and 0 for the smaller ones.   

\begin{center}

  \begin{tabular}{@{} ccc@{} }
    \includegraphics[width=40mm]{corr_matrix_r_000.eps} &
    \includegraphics[width=40mm]{corr_matrix_r_026.eps} &
    \includegraphics[width=40mm]{corr_matrix_r_038.eps} \\
    \includegraphics[width=40mm]{corr_matrix_r_050.eps} &
    \includegraphics[width=40mm]{corr_matrix_r_068.eps} &
    \includegraphics[width=40mm]{corr_matrix_r_080.eps} \\
  \end{tabular}


\begin{footnotesize}
 Figure 17, $r$ given on top of subfigures stands for the threshold values applied to the functional connectivity matrix $f_{ij}$ (here $f_{ij}$ corresponds to $A.txt$). Both of the x and y axis have scales from 0 to 64. 
\end{footnotesize}
\end{center}

\section{Isolated FitzHugh-Nagumo Neural Model}

$[VUK13]$ paper simulates the neural network dynamics of one node with $u_i$ (attractor) and $v_i$ (inhibitor) by using FHN model as in the following eqautions; 

\begin{equation}
 \dot{u}_i=g(u_i,v_i)-c \sum_{j=1}^N  f_{ij} u_j(t-\Delta t_{ij})+n_u
\end{equation}
\begin{equation}
 \dot{v}_i=h(u_i,v_i)+n_v
\end{equation}
where $c$ is coupling strength, $f_{ij}$ is the connectivity matrix, $\Delta t_{ij}$ is time delay due to finite signal propagation velocity between nodes, $n_u$ is the noise factor. $\Delta t_{ij}$ is calculated as $\Delta t_{ij}=\frac{d_{ij}}{\nu}$, distance matrix divided by velocity and noise factor is includes the noise strength $D$. Note that $i,j=0,1,2,...,N$

The functions $g$ and $v$ are modeled very similar to FitzHugh-Nagumo model introduced before:
\begin{equation}
 \dot{u}=g(u,v)=\tau(v+\gamma u - \frac{u^3}{3})
\end{equation}
\begin{equation}
 \dot{v}=h(u,v)=-\frac{1}{\tau}(u- \alpha +bv-I)
\end{equation}
This section aims to observe the noise strength $D$ on an isolated system meaning no coupling (simply by $c=0$). Figures below indicate the attractor behavior of the first node $u_1$ over time with different noise strengths. 

\begin{center}
  \begin{tabular}{@{} c@{} }
    \includegraphics[width=120mm]{u_1_r_050_c_0_D_0.eps} \\
    \includegraphics[width=120mm]{u_1_r_050_c_0_D_001.eps} \\
    \includegraphics[width=120mm]{u_1_r_050_c_0_D_005.eps} \\
    \includegraphics[width=120mm]{u_1_r_050_c_0_D_010.eps} \\
  \end{tabular}
	\newline

\begin{footnotesize}
 Figure 18, Time evolution of $u_1$ when $c=0$, the noise strengths from top to below: $D=0$, $D=0.01$, $D=0.05$, $D=0.1$ ( $I=0$, $v=7m/s$, $b=0.2$, $\tau=1.25$, $\alpha=0.85$, $\gamma=1$, $r=0.5$)
\end{footnotesize}
\end{center}


Now, let us have a look at the attractor behavior of all the isolated nodes $u_i(t)$, $i=1,2,3,...,N$ in $t=500ms$ with different noise strengths. (?=compare with section 1 and stability of figures below) 

\begin{center}
  \begin{tabular}{@{} c@{} }
    \includegraphics[width=80mm,height=40mm]{sample_A_r0_5_sigma=0_0_D=0_0_v=70_0_tmax=1000.eps} \\
    \includegraphics[width=80mm,height=40mm]{sample_A_r0_5_sigma=0_0_D=0_01_v=70_0_tmax=1000.eps} \\
    \includegraphics[width=80mm,height=40mm]{sample_A_r0_5_sigma=0_0_D=0_05_v=70_0_tmax=1000.eps} \\
    \includegraphics[width=80mm,height=40mm]{sample_A_r0_5_sigma=0_0_D=0_1_v=70_0_tmax=1000.eps} \\
  \end{tabular}
	\newline

\begin{footnotesize}
 Figure 19, Time evolution of $u_i(t)$ when $c=0$, the noise strengths from top to below: $D=0$, $D=0.01$, $D=0.05$, $D=0.1$. ($I=0$, $v=7m/s$, $b=0.2$, $\tau=1.25$, $\alpha=0.85$, $\gamma=1$, $r=0.5$)
\end{footnotesize}
\end{center}

\section{Distance Distribution between Cortical Regions}
\begin{center}
    \includegraphics[width=\textwidth,height=80mm]{P(d)_distance_FSL.eps} 
	\newline
\begin{footnotesize}
 Figure 20, Distance distribution between the $N=64$ nodes, source: $d_{ij}=$\textit{FSL\_ROIs\_distance\_matrix.dat}, the distances between nodes are Euclidean.
\end{footnotesize}
\end{center}

\begin{center}
    \includegraphics[width=\textwidth,height=80mm]{P(d)_distance_FSL_colorbar.eps} 
	\newline
\begin{footnotesize}
 Figure 21, Euclidean distance matrix $d_{ij}$ in color code
\end{footnotesize}
\end{center}

\section{Correlation Distribution of fMRI Data}

\begin{center}
    \includegraphics[width=\textwidth,height=80mm]{P(d)_distance_A_txt.eps} 
	\newline
\begin{footnotesize}
 Figure 22, Data distribution in functional connectivity matrix as a result of fMRI signaling, $f_{ij}$=\textit{A.txt}, where ${i,j}=1,2,...,N=64$
\end{footnotesize}
\end{center}

\begin{center}
    \includegraphics[width=\textwidth,height=80mm]{colorbar_A_r_000.eps} 
	\newline
\begin{footnotesize}
 Figure 23, Functional connectivity matrix $f_{ij}$ in color code. $f_{ij}$=\textit{A.txt}, there is no threshold applied on it, $r=0$
\end{footnotesize}
\end{center}


\begin{center}

  \begin{tabular}{@{} ccc@{} }
    
    \includegraphics[width=45mm,height=45mm]{colorbar_A_r_026.eps} &
\includegraphics[width=45mm,height=45mm]{colorbar_A_r_038.eps} &  	\includegraphics[width=45mm,height=45mm]{colorbar_A_r_050.eps} \\

  \end{tabular}


\begin{footnotesize}
 Figure 24, Functional connectivity matrices $f_{ij}$ with applied thresholds in color code, red color for the ones, and blue for the zeros. Figure on the left: $f_{ij}$=\textit{A\_r.0.26.dat}, in the middle: $f_{ij}$=\textit{A\_r.0.38.dat}, on the right: $f_{ij}$=\textit{A\_r.0.50.dat} 
\end{footnotesize}
\end{center}

\section{Visualization of $f_{ij}$ in 2D Anatomical Space with different $r$s}
\begin{center}
  \begin{tabular}{@{} cc@{} }
    \includegraphics[width=70mm]{anatomic_A_r_000.eps} &
    \includegraphics[width=70mm]{anatomic_A_r_026.eps} \\
    \includegraphics[width=70mm]{anatomic_A_r_038.eps} &
    \includegraphics[width=70mm]{anatomic_A_r_050.eps} \\
  \end{tabular}
\begin{footnotesize}
 Figure 25, Visualization of threshold matrices in anatomical space by locating each region according to its $x$ and $y$ coordinates and drawing a link between significantly connected regions. (?= why r=0 is similar to r=0.5)
\end{footnotesize}
\end{center}

\section{Visualization of $f_{ij}$ in 3D Anatomical Space with different $r$s}
\begin{center}
  \begin{tabular}{@{} cc@{} }
    \includegraphics[width=60mm]{anatomic_3D_A_r_000.eps} &
    \includegraphics[width=60mm]{anatomic_3D_A_r_026.eps} \\
    \includegraphics[width=60mm]{anatomic_3D_A_r_038.eps} &
    \includegraphics[width=60mm]{anatomic_3D_A_r_050.eps} \\
  \end{tabular}
\begin{footnotesize}
 Figure 26, Visualization of threshold matrices in anatomical space by locating each region according to its $x$, $y$ and $z$ coordinates and drawing a link between significantly connected regions. (?= why r=0 is similar to r=0.5)
\end{footnotesize}
\end{center}

\textit{This part is not yet completed.}


\section{Simulated Bold Signals - The new $u_i$ time series with Baloon-Windkessel Model }
\begin{center}
  \begin{tabular}{@{} c@{} }
    \includegraphics[width=120mm]{simul_u_1_r_050_c_0_D_0.eps} \\
    \includegraphics[width=120mm]{simul_u_1_r_050_c_0_D_001.eps} \\
    \includegraphics[width=120mm]{simul_u_1_r_050_c_0_D_005.eps} \\
  \end{tabular}
\begin{footnotesize}
 Figure 27, Time evolution of $u_1$ when $c=0$, the noise strengths from top to below: $D=0$, $D=0.01$, $D=0.05$, $D=0.1$ ( $I=0$, $v=7m/s$, $b=0.2$, $\tau=1.25$, $\alpha=0.85$, $\gamma=1$, $r=0.5$)
\end{footnotesize}
\end{center}

\begin{itemize}
 \item Discuss difference between Figure 25 and Figure 18 and shorter time now(?)
 \item When all the $u_i$ series plotted over time, it does not seem clearly different than the $u_1(t)$ evolution. (?)
\end{itemize} 

When "Butterworth lowpass filter of order 5" is applied to the simulated BOLD signalling, then the $u_i$ time series look like as in the following figures.  

\begin{center}
  \begin{tabular}{@{} c@{} }
    \includegraphics[width=120mm]{simul_filt_u_1_r_050_c_0_D_0.eps} \\
    \includegraphics[width=120mm]{simul_filt_u_1_r_050_c_0_D_001.eps} \\
    \includegraphics[width=120mm]{simul_filt_u_1_r_050_c_0_D_005.eps} \\
  \end{tabular}
\begin{footnotesize}
 Figure 28, Time evolution of $u_i$ when $c=0$, the noise strengths from top to below: $D=0$, $D=0.01$, $D=0.05$, $D=0.1$ ( $I=0$, $v=7m/s$, $b=0.2$, $\tau=1.25$, $\alpha=0.85$, $\gamma=1$, $r=0.5$)
\end{footnotesize}
\end{center}

\begin{itemize}
 \item Discuss filtering with Vesna (?)
\end{itemize} 

\newpage
\section{To be continued... FHN Model}
How to change parameters in [VUK] paper in order to get similar state space graph as in the extenden FHN model? Answer: 
\begin{lstlisting}
gamma=0.9;  		%gamma close to 1
b=-0.2;    		%change + incline to - incline 
x_limit=2.5;
y_limit=2;
xE=(-x_limit:0.01:x_limit);
yE1=-xE.^3/3 + gamma*xE;  % make yE totally minus
yE2=(alpha-xE)/b;
 \end{lstlisting}


\begin{center}
  
    \includegraphics[width=\textwidth]{deneme.eps} \\
 
\begin{footnotesize}
 Figure 29
\end{footnotesize}
\end{center}

\textit{This part is not yet comleted.}


\end{document}
